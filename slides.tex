\documentclass{hsharz-slides}

%set size of din a 4

\setlength{\paperwidth}{29.7cm}
\setlength{\paperheight}{21.0cm}
\setlength{\textwidth}{27cm}
\setlength{\textheight}{20.0cm} 

\title       {Titel der Präsentation
            \\in 38 Pt Bold }
\subtitle    {Der Subtitel}
\author      {Vorname Nachname}
\newcommand{\function}{Funktion des Vortragenen}
%Akutelles Datum oder Datum selbst setzen, dann auskommentieren 
%\date     {17.07.2012}

%%%%%%%%%%%%%%%%%%%%%%%%%%%%%%%%%%%%%%%%
% Die logos können hier festgelegt werden
%%%%%%%%%%%%%%%%%%%%%%%%%%%%%%%%%%%%%%%%

\newcommand{\hslogo}{images/logo/HSH-Logo-RGB-dt.png}
\newcommand{\hslogoblack}{images/logo/HSH-Logo-RGB-dt-black.pdf}

%%%%%%%%%%%%%%%%%%%%%%%%%%%%%%%%%%%%%%%%
% colorAI für den Fachbereich AI
% colorVW für den Fachbereich VW
% colorWW für den Fachbereich WW
%%%%%%%%%%%%%%%%%%%%%%%%%%%%%%%%%%%%%%%%
\newcommand{\ColorTheme}{colorAI}

\begin{document}
    \setbeamercolor{background canvas}{bg=\ColorTheme}
    
    \begin{frame}
        \titlepage  
    \end{frame}
    
    \setbeamercolor{background canvas}{bg=}

    	
	\begin{frame}
		\frametitle{Aufzählung}
		\framesubtitle{Optionale Unterüberschrift}		
		Beispieltext in 20 Pt
		\begin{itemize}
			\item \lorem
			\item \lorem
			\item \lorem
			\item \lorem
			\item \lorem
			\item \lorem
			\item \lorem
			\item \lorem
			\item \lorem
		\end{itemize}
	\end{frame}
	
	\begin{frame}
	    \begin{block}{Block title} 
	        This is a block in blue
        \end{block}
        \begin{alertblock}{Alert-block title}
	        This is a block in red
        \end{alertblock}
	\end{frame}
    \begin{frame}
        \frametitle{Spalten}
		\framesubtitle{eine Folie mit 2 Spalten}	
        \begin{columns}
            \column{.55\textwidth}
                \pgfimage[width=\textwidth]{images/tasse}
            \column{.45\textwidth}
                \begin{itemize}
			        \item \loremshort
			        \item \loremshort
	           	 \item \loremshort
		\end{itemize}

        \end{columns}
    \end{frame}
	
	\begin{frame}
		\frametitle{boxes}
		
        \begin{beamerboxesrounded}[upper= uppercol ,shadow=false]
            {Definition - What is A:} 
            $A:= 2 + 5$.
        \end{beamerboxesrounded}
	\end{frame}
	
	    
    \begin{frame}{Quellen}
        
        
    \end{frame}

\end{document}
